\documentclass{article}
\title{Technical Documentation Avionics Bay\\Elpis MK IIb}
\usepackage{amsmath}
\usepackage{amssymb}
\usepackage{pgfplots}
\usepackage{unicode-math}
\usepackage{emoji}
\usepackage[american]{circuitikz}
\usepackage{enumitem}
\usepackage{float}
\usepgfplotslibrary{fillbetween}
\pgfplotsset{compat = newest}
\date{}
\author{Mytikas}

\newcommand{\twopartdef}[4]
{
	\left\{
		\begin{array}{ll}
			#1 & \mbox{if } #2 \\
			#3 & \mbox{if } #4
		\end{array}
	\right. %}
}
\newcommand{\pd}[2]
{
	\frac{\partial#1}{\partial#2}
}

\begin{document}
\maketitle
\section*{Hardware}
\subsection*{Components}
\begin{table}[h]
\begin{tabular}{ll}
Microcontroller & Arduino MKR Zero \\
Barometer & BMP 280 \\
Accelerometer & BNO 085 \\
Servo & \\
\end{tabular}
\end{table}

\subsection*{Circuit diagram}
\begin{figure}[!h]
\centering
\scalebox{0.7}{
\begin{circuitikz}
\def\sensor(#1)#2#3{
	\begin{scope}[shift={(#1)}]
	\draw (-1.5,-2.5) rectangle (1.5,2.5)
	(0,-0.25) node[anchor=south] {#2}
	(-1.5, 1.5) node[anchor=west] {$V_{cc}$} coordinate (#3 vcc)
	(-1.5, 0.5) node[anchor=west] {GND} coordinate (#3 gnd)
	(-1.5,-0.5) node[anchor=west] {SCL/SCK} coordinate (#3 scl)
	(-1.5,-1.5) node[anchor=west] {SDA/SDI} coordinate (#3 sda)
	;
	\end{scope}
}

\sensor(5.5,0){BMP-280}{bar}
\sensor(5.5,-6){BNO-085}{acc}
\draw(-10,2.5) rectangle (-5, -8) (-7.5, -2.5) node[anchor=south] {Arduino MKR Zero};
\draw(-2, -7) rectangle (1, -8) (-0.5, -7.5) node[anchor=north] {Servo};
\draw
(-5, 1.5) node[anchor=east] {$V_{cc}$} -- (bar vcc)
(-5, 1.5) to[short, -*] ++(8, 0) |- (acc vcc)
(-5, 0.5) node[anchor=east] {GND} -- (bar gnd)
(-5, 0.5) to[short, -*] ++(7.5, 0) |- (acc gnd)
(-5, -0.5) node[anchor=east] {12 SCL} -- (bar scl)
(-5, -0.5) to[short, -*] ++(7, 0) |- (acc scl)
(-5, -1.5) node[anchor=east] {11 SDA} -- (bar sda)
(-5, -1.5) to[short, -*] ++(6.5, 0) |- (acc sda)

(-5, -7.5) node[anchor=east] {9 SCK} -- (-2, -7.5) node[anchor=west] {Ctrl}
(-1, 1.5) to[short, *-] (-1, -7) node[anchor=north] {Pwr} 
(0, 0.5) to[short, *-] (0, -7) node[anchor=north] {Gnd} 

(-5, -3.5) node[anchor=east] {$\sim2$} to[leDo, -*] ++(3,0)
(-5, -4.5) node[anchor=east] {$\sim3$} to[leDo, -*] ++(3,0)
(-5, -5.5) node[anchor=east] {$\sim4$} to[leDo, -*] ++(3,0)
(-5, -6.5) node[anchor=east] {$\sim5$} to[leDo] ++(3,0) -- ++(0, 3) to[R=220 $\Omega$] ++(0, 2) to[short, -*] ++(0,2)
;
\end{circuitikz}
}
\end{figure}
\subsection*{Pinmap}
\begin{table}[H]
\begin{tabular}{llll}
Printed name & Compiler name & Variable name & Use \\
\hline
$V_{cc}$ & n/a & n/a & Power supply voltage \\
GND & n/a & n/a & Ground \\
12 SCL & n/a & n/a & Clock signal for I$^2$C \\
11 SDA & n/a & n/a & Data signal for I$^2$C \\
9 SCK & 9 & SERVO\_PIN & Signal for parachute servo \\
$\sim2$ & 2 & LED\_PIN & Status LED \\
$\sim3$ & 3 & ERROR\_LED\_PIN & Error LED \\
$\sim4$ & 4 & LAUNCH\_LED\_PIN & Launch LED \\
$\sim5$ & 5 & CARD\_LED\_PIN & Card LED \\
\end{tabular}
\end{table}

\section*{Software}

\subsection{Accelerometer class}

\subsubsection*{Public Interface}
The Accelerometer class's public interface has a constructor and a \verb|getData()| method. The contructor initializes the accelerometer, and sets the appropriate values.
the getData method takes a pointer to a telemetry struct as an argument. It reads values from the accelerometer (Bno085), processes them and writes them to the \verb|telemetry| instance.

\subsubsection*{Private parts}
\paragraph*{Constructor}
The constructor begins by initializing most of the members. It then repeatedly tries to start the BNO, and lights the Error LED during failure. It then repeatedly attempts
to set the desired reports. If it were to fail, the Error LED is lit until success.

\paragraph*{getData()}
\verb|getData()| starts out by checking if the BNO was reset. If it was, then the reports are set again. It also logs the reset in the \verb|telemetry| instance. Then it atempts to read the 
reports in a while loop. After that it checks if a read value is zero. This happens sometimes, we don't know why. If it is zero, then the last nonzero value is put in its place.
This is why the next step is writing the \verb|trot| and \verb|tacc| values to \verb|acc| and \verb|rot| members. \verb|acc| and \verb|rot| are then written to the \verb|telemetry| instance.
A vector \verb|racc| is then created by rotating \verb|acc| by \verb|rot|. It is also written to the \verb|telemetry| instance.

\subsection{Vector Types}
There are currently two vector types, quaternions(\verb|Quat|) and three dimensional vectors(\verb|Vec3|).


\subsubsection*{Quaternion}
A quaternion is a hypercomplex number. This means it has four components; one real, and three imaginary(i, j, k).
\begin{equation}
	Q = a + b\mathbf{i} + c\mathbf{j} + d\mathbf{k}
\end{equation}

As for normal complex number there are special arithmetic rules for the imaginary parts. Addition is done component-wise, multiplication is done using a
Hamilton product(see muliply). Quaternions are often used to represent a rotation in 3d-space, and that is what they are doing in this codebase. A rotation of $\theta$ degrees around
the axis (x,y,z) would look like
\begin{equation}
	Q=cos\frac{\theta}{2} + xsin\frac{\theta}{2}\mathbf{i} + ysin\frac{\theta}{2}\mathbf{j} + zsin\frac{\theta}{2}\mathbf{k}
\end{equation}
as a quaternion.

\paragraph*{Constructors}
There are two constructors, one without arguments, which returns a zero \verb|quat|, and one with four arguments. The four arguments are one for each component of the vector,
and sets the members to these values.

\paragraph*{print()}
There is a print function implemented for \verb|quat|s. The first argument is the name of the vector. This will be printed out before the values. The second argument determines
whether or not to put a linebreak at the end. The call \verb|Quat().print("Rotation", false)| will print\\
\verb|Rotation: r: 0.0, i: 0.0, j: 0.0, k: 0.0|.

\paragraph*{invert()}
This functions inverts the \verb|quat|. This is the same as taking the complex conjugate of the number, that is, negating all imaginary parts:
\begin{equation}
	\Bar{Q} = a - b\mathbf{i} - c\mathbf{j} - d\mathbf{k}
\end{equation}

\paragraph*{mulitply()}
This returns the Hamilton product of two quaternions. This product is not commutative, that is $Q_1*Q_2 \neq Q_2*Q_1$. This functions takes an argument \verb|q2| and multiplies
it onto the object from the right, meaning that \verb|q1.muliply(q2)| is mathematically eqiavalent to $q1*q2$, and not $q2*q1$. Mathematically, the product looks like this:
\begin{eqnarray*}
	Q_1 = a_1 + b_1\mathbf{i} + c_1\mathbf{j} + d_1\mathbf{k}\\
	Q_2 = a_2 + b_2\mathbf{i} + c_2\mathbf{j} + d_2\mathbf{k}\\
	Q_1*Q_2 = a_1a_2 -b_1b_2 - c_1c_2 - d_1d_2\\
			+ (a_1b_2 + b_1a_2 + c_1d_2 - d_1c_2)\mathbf{i}\\
			+ (a_1c_2 - b_1d_2 + c_1a_2 + d_1b_2)\mathbf{j}\\
			+ (a_1d_2 + b_1c_2 - c_1b_2 + d_1a_2)\mathbf{k}\\
\end{eqnarray*}
This represents the rotation $Q_1$ followed by the rotation $Q_2$, if both $Q_1$ and $Q_2$ are rotation Quaternions.

\subsubsection*{Vector 3}
\paragraph*{Constructors}
\verb|Vec3| has two constructors, an empty one which returns the zero vector, and one with three arguments for the x, y and z components respectively.

\paragraph*{print()}
Prints out the \verb|Vec3| to Serial. The first argument is the name of the vector and will be printed before the values. The second one detmermines whether to break the line
at the end. The call\\ 
\verb|Vec3().print("Acceleration", false)| prints \\
\verb|Acceleration: x: 0.0, y: 0.0, z: 0.0|

\paragraph*{rotate()}
The rotate function rotates the \verb|Vec3| object with a quaternion. Mathematically, rotating the vector V with Q would look like this:
\begin{eqnarray}
	Q_V = 0.0 + V_x\mathbf{i} + V_y\mathbf{j} + V_z\mathbf{k}\\
	Q_R = Q*Q_V*\Bar{Q}\\
\end{eqnarray}
Here the imaginary components of $Q_R$ are the x, y and z components of the rotated V. The multiplication sign means the Hamilton product.

\end{document}